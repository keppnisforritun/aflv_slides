\documentclass{beamer}
\usefonttheme[onlymath]{serif}
\usepackage[T1]{fontenc}
\usepackage[utf8]{inputenc}
\usepackage[english]{babel}
\usepackage{amsmath}
\usepackage{amssymb}
\usepackage{amsthm}
\usepackage{gensymb}
\usepackage{parskip}
\usepackage{mathtools}
\usepackage{listings}
\usepackage{hyperref}
\usepackage{graphicx}
\usepackage{color}
\usepackage{enumerate}
\usepackage{tikz}
\usetikzlibrary{calc}
\usetikzlibrary{positioning}
\usetikzlibrary{angles}
\usetikzlibrary{shapes}
\usetikzlibrary{arrows}
\usepackage{verbatim}
\usepackage{multicol}
\usepackage{array}
\usepackage{minted}
\parskip 0pt


\DeclareMathOperator{\lcm}{lcm}
\newcommand\floor[1]{\left\lfloor#1\right\rfloor}
\newcommand\ceil[1]{\left\lceil#1\right\rceil}
\newcommand\abs[1]{\left|#1\right|}
\newcommand\p[1]{\left(#1\right)}
\newcommand\sqp[1]{\left[#1\right]}
\newcommand\cp[1]{\left\{#1\right\}}
\newcommand\norm[1]{\left\lVert#1\right\rVert}
\renewcommand\Im{\operatorname{Im}}
\renewcommand\Re{\operatorname{Re}}

\usetheme{metropolis}
\definecolor{dark yellow}{rgb} {0.6,0.6,0.0}
\definecolor{dark green}{rgb} {0.0,0.6,0.0}

\graphicspath{{myndir/}}

\title{Complete Search}
\author{Arnar Bjarni Arnarson \& Atli FF}
\institute{\href{http://ru.is/td}{School of Computer Science} \\[2pt] \href{http://ru.is}{Reykjavík University}}
\titlegraphic{\hfill\includegraphics[height=0.6cm]{kattis}}

\begin{document}
\maketitle

\begin{frame}[plain]{Complete search}
    \begin{itemize}
        \item We have a finite set of objects
        \item We want to find an element in that set which satisfies some constraints
        \begin{itemize}
            \item or find \textbf{all} elements in that set which satisfy some constraints
        \end{itemize}

        \vspace{5pt}
        \item Simple! Just go through all elements in the set, and for each of them check if they satisify the constraints
        \item Of course it's not going to be very efficient...
        \item But remember, we always want the simplest solution that runs in time
        \item Complete search should be the first problem solving paradigm you think about when you're trying to solve a problem
    \end{itemize}
\end{frame}

\begin{frame}[plain]{Example problem: Closest Sums}
    \begin{itemize}
        \item https://open.kattis.com/problems/closestsums
    \end{itemize}
\end{frame}

\begin{frame}[plain]{Complete search}
    \begin{itemize}
        \item What if the search space is more complex?
            \begin{itemize}
                \item All permutations of $n$ items
                \item All subsets of $n$ items
                \item All ways to put $n$ queens on an $n\times n$ chessboard without any queen attacking any other queen
            \end{itemize}
        \item How are we supposed to iterate through the search space?
        \item Let's take a better look at these examples
    \end{itemize}
\end{frame}

\begin{frame}[plain,fragile]{Iterating through permutations}
    \begin{itemize}
        \item Already implemented in many standard libraries:
            \begin{itemize}
                \item \texttt{next\_{}permutation} in C++
                \item \texttt{itertools.permutations} in Python
            \end{itemize}
    \end{itemize}

            \begin{minted}{cpp}
int n = 5;
vector<int> perm(n);
for (int i = 0; i < n; i++) perm[i] = i + 1;

do {
    for (int i = 0; i < n; i++) {
        printf("%d ", perm[i]);
    }
    printf("\n");

} while (next_permutation(perm.begin(), perm.end()));
            \end{minted}

\end{frame}

\begin{frame}[plain]{Iterating through permutations}
    \begin{itemize}
        \item Even simpler in Python
        \vspace{20pt}
        \item Remember that there are $n!$ permutations of length $n$, so usually you can only go through all permutations if $n \leq 10$
            \begin{itemize}
                \item Otherwise you need to find a more clever approach than complete search
            \end{itemize}
            \vspace{20pt}
    \end{itemize}
\end{frame}

\begin{frame}[plain]{Example problem: Veci}
    \begin{itemize}
        \item https://open.kattis.com/problems/veci
    \end{itemize}
\end{frame}

\begin{frame}[plain,fragile]{Iterating through subsets}
    \begin{itemize}
        \item Remember the bit representation of subsets?
        \item Each integer from $0$ to $2^n - 1$ represents a different subset of the set $\{1,2,\ldots,n\}$
        \item Just iterate through the integers
    \end{itemize}

            \begin{minted}{cpp}
int n = 5;
for (int subset = 0; subset < (1 << n); subset++) {
    for (int i = 0; i < n; i++) {
        if ((subset & (1 << i)) != 0) {
            printf("%d ", i+1);
        }
    }
    printf("\n");
}
            \end{minted}
\end{frame}

\begin{frame}[plain]{Iterating through subsets}
    \begin{itemize}
        \item Similar in Python
        \vspace{20pt}
        \item Remember that there are $2^n$ subsets of $n$ elements, so usually you can only go through all subsets if $n \leq 25$
            \begin{itemize}
                \item Otherwise you need to find a more clever approach than complete search
            \end{itemize}
            \vspace{20pt}
    \end{itemize}
\end{frame}

\begin{frame}[plain]{Example problem: Exam Manipulation}
    \begin{itemize}
        \item https://open.kattis.com/problems/exammanipulation
    \end{itemize}
\end{frame}

\begin{frame}[plain,fragile]{Backtracking}
    \begin{itemize}
        \item We've seen two ways to go through a complex search space, but both of the solutions were rather specific
        \item Would be nice to have a more general ``framework''
        \vspace{10pt}
        \item Backtracking!
    \end{itemize}
\end{frame}

\section*{Backtracking}

\begin{frame}[plain,fragile]{Backtracking}
    \begin{itemize}
        \item Define states
            \begin{itemize}
                \item We have one initial ``empty'' state
                \item Some states are partial
                \item Some states are complete
            \end{itemize}
        \vspace{10pt}
        \item Define transitions from a state to possible next states
        \vspace{10pt}
        \item Basic idea:
            \begin{enumerate}
                \item Start with the empty state
                \item Use recursion to traverse all states by going through the transitions
                \item If the current state is invalid, then stop exploring this branch
                \item Process all complete states (these are the states we're looking for)
            \end{enumerate}
    \end{itemize}
\end{frame}


\begin{frame}[plain,fragile]{Backtracking}
    \begin{itemize}
        \item General solution form:
    \end{itemize}

    \begin{minted}[fontsize=\scriptsize]{cpp}
state S;

void generate() {
    if (!is_valid(S))
        return;

    if (is_complete(S))
        print(S);

    foreach (possible next move P) {
        apply move P;
        generate();
        undo move P;
    }
}

S = empty state;
generate();
    \end{minted}
\end{frame}


\begin{frame}[plain,fragile]{Generating all subsets}
    \begin{itemize}
        \item Also simple to do with backtracking:
    \end{itemize}
    \begin{minted}[fontsize=\tiny]{cpp}
const int n = 5;
bool pick[n];

void generate(int at) {
    if (at == n) {
        for (int i = 0; i < n; i++) {
            if (pick[i]) {
                printf("%d ", i+1);
            }
        }
        printf("\n");
    } else {

        // either pick element no. at
        pick[at] = true;
        generate(at + 1);

        // or don't pick element no. at
        pick[at] = false;
        generate(at + 1);
    }
}

generate(0);
    \end{minted}
\end{frame}


\begin{frame}[plain,fragile]{Generating all permutations}
    \begin{itemize}
        \item Also simple to do with backtracking:
    \end{itemize}
    \begin{minted}[fontsize=\tiny]{cpp}
const int n = 5;
int perm[n];
bool used[n];

void generate(int at) {
    if (at == n) {
        for (int i = 0; i < n; i++) {
            printf("%d ", perm[i]+1);
        }
        printf("\n");
    } else {

        // decide what the at-th element should be
        for (int i = 0; i < n; i++) {
            if (!used[i]) {
                used[i] = true;
                perm[at] = i;

                generate(at + 1);

                // remember to undo the move:
                used[i] = false;
            }
        }
    }
}

memset(used, 0, n);
generate(0);
    \end{minted}
\end{frame}

\begin{frame}[plain,fragile]{$n$ queens}
    \begin{itemize}
        \item Given $n$ queens and an $n\times n$ chessboard, find all ways to
            put the $n$ queens on the chessboard such that no queen can attack
            any other queen

        \vspace{10pt}

        \item This is a very specific set we want to iterate through, so we probably won't find this in the standard library
        \item We could use our bit trick to iterate through all subsets of the $n\times n$ cells of size $n$, but that would be very slow
        \vspace{20pt}
        \item Let's use backtracking
    \end{itemize}
\end{frame}

\begin{frame}[plain,fragile]{$n$ queens}
    \begin{itemize}
        \item Go through the cells in increasing order
        \item Either put a queen on that cell or not (transition)
        \item Don't put down a queen if she's able to attack another queen already on the table
    \end{itemize}

    \vspace{10pt}

    \begin{minted}{cpp}
const int n = 8;
bool has_queen[n][n];
int queens_left = n;

// generate function

memset(has_queen, 0, sizeof(has_queen));
generate(0, 0);
    \end{minted}
\end{frame}

\begin{frame}[plain,fragile]{$n$ queens}
    \begin{minted}[fontsize=\tiny]{cpp}
void generate(int x, int y) {
    if (y == n) {
        generate(x+1, 0);
    } else if (x == n) {
        if (queens_left == 0) {
            for (int i = 0; i < n; i++) {
                for (int j = 0; j < n; j++) {
                    printf("%c", has_queen[i][j] ? 'Q' : '.');
                }
                printf("\n");
            }
        }
    } else {

        if (queens_left > 0 and no queen can attack cell (x,y)) {
            // try putting a queen on this cell
            has_queen[x][y] = true;
            queens_left--;

            generate(x, y+1);

            // undo the move
            has_queen[x][y] = false;
            queens_left++;
        }

        // try leaving this cell empty
        generate(x, y+1);
    }
}
    \end{minted}
\end{frame}

\begin{frame}[plain]{Example problem: Lucky Numbers}
    \begin{itemize}
        \item https://open.kattis.com/problems/luckynumber
        \item As a note, another classic use case is solving sudokus
    \end{itemize}
\end{frame}

\section*{Meet in the middle}

\begin{frame}[plain,fragile]{Halving the exponent}
    \begin{itemize}
        \item We now know at least two different ways we could solve the subset sum problem, i.e. given some numbers check if there is a subset summing to some given target value
        \item They would have time complexity $\mathcal{O}(2^n)$ or even $\mathcal{O}(n2^n)$, which is quite a lot.
        \item But there is a trick we can employ to get $\mathcal{O}(n2^{n/2})$ instead.
    \end{itemize}
\end{frame}

\begin{frame}[plain,fragile]{Halves}
    \begin{itemize}
        \item Split the numbers into two groups $A, B$, then iterate over every subset of $A$ and put it into a set. This takes $\mathcal{O}(n2^{n/2})$ time.
        \item Then iterate over every subset in $B$, and each time check if the target value minus the sum of $B$ is in our set of sums. This also takes $\mathcal{O}(n2^{n/2})$ time.
        \item This way we do the problem one half at a time, and meet in the middle.
    \end{itemize}
\end{frame}

\begin{frame}[plain,fragile]{Middle}
    \begin{itemize}
        \item This can be applied more generally, searching from both ends of a problem.
        \item It is a fairly common tactic in cryptography as well.
        \item Sometimes it can also be combined with backtracking, backtracking from two ends at once.
    \end{itemize}
\end{frame}

\begin{frame}[plain,fragile]{Not quite half}
    \begin{itemize}
        \item Sometimes the gain isn't quite a square root of the base of the exponential run-time, but gains can still be large.
        \item Take the problem of having a set of numbers $\{x_1, \dots, x_n\}$ and wanting to find two different non-empty subsets $\{x_{i_1}, \dots, x_{i_r}\}$ and $\{x_{j_1}, \dots, x_{j_s}\}$ with the same sum.
        \item The straight forward way would be $\mathcal{O}(4^n)$, but iterating over all subsets and counting how many get to any given sum is $\mathcal{O}(n2^n)$.
    \end{itemize}
\end{frame}

\begin{frame}[plain,fragile]{Not quite half}
    \begin{itemize}
        \item But we can split $x_1, \dots, x_n$ in two halves, then rewrite the condition
        \[ x_{i_1} + \dots+ x_{i_r} = x_{j_1} + \dots + x_{j_s}\]
        as
        \[ x_{i_1} + \dots + x_{i_r'} - x_{j_1} - \dots - x_{j_s'} = x_{j_s'+1} + \dots + x_{j_s} - x_{i_r' + 1} - \dots - x_{i_r}\]
        so that we move every term in the first half to the left hand side and every term in the second half to the right side.
        \item Then we can meet in the middle on all ways to ignore, add or subtract the element from the sum to get a $\mathcal{O}(n3^{n/2})$.
    \end{itemize}
\end{frame}

\end{document}

