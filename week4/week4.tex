\documentclass{beamer}
\usefonttheme[onlymath]{serif}
\usepackage[T1]{fontenc}
\usepackage[utf8]{inputenc}
\usepackage[english]{babel}
\usepackage{amsmath}
\usepackage{amssymb}
\usepackage{amsthm}
\usepackage{gensymb}
\usepackage{parskip}
\usepackage{mathtools}
\usepackage{listings}
\usepackage{hyperref}
\usepackage{graphicx}
\usepackage{color}
\usepackage{enumerate}
\usepackage{tikz}
\usetikzlibrary{calc}
\usetikzlibrary{positioning}
\usetikzlibrary{angles}
\usetikzlibrary{shapes}
\usetikzlibrary{arrows}
\usepackage{verbatim}
\usepackage{multicol}
\usepackage{array}
\usepackage{minted}
\parskip 0pt


\DeclareMathOperator{\lcm}{lcm}
\newcommand\floor[1]{\left\lfloor#1\right\rfloor}
\newcommand\ceil[1]{\left\lceil#1\right\rceil}
\newcommand\abs[1]{\left|#1\right|}
\newcommand\p[1]{\left(#1\right)}
\newcommand\sqp[1]{\left[#1\right]}
\newcommand\cp[1]{\left\{#1\right\}}
\newcommand\norm[1]{\left\lVert#1\right\rVert}
\renewcommand\Im{\operatorname{Im}}
\renewcommand\Re{\operatorname{Re}}

\usetheme{metropolis}
\definecolor{dark yellow}{rgb} {0.6,0.6,0.0}
\definecolor{dark green}{rgb} {0.0,0.6,0.0}

\graphicspath{{myndir/}}

\title{Greedy Algorithms}
\author{Atli FF}
\institute{\href{http://ru.is/td}{School of Computer Science} \\[2pt] \href{http://ru.is}{Reykjavík University}}
\titlegraphic{\hfill\includegraphics[height=0.6cm]{kattis}}

\begin{document}
\maketitle

\begin{frame}[plain]{Greedy algorithms}
    \begin{itemize}
        \item An algorithm that always makes \textit{locally} optimal moves is called greedy
        \item For some kinds of problems this will give a \textit{globally} optimal solution as well
        \item Seeing when this is the case can be very tricky, and if used in the wrong context the solution will get a WA verdict
    \end{itemize}
\end{frame}

\begin{frame}[plain]{Submitting greedy solutions}
    \begin{itemize}
        \item The tricky thing with these solutions are that it's often hard to know if you've made a mistake and thus get WA or if there's some hole in the greedy algorithm
        \item It's often easy to think of all kinds of greedy solutions, but they are very often wrong
        \item Generally one would like to consider complete search or dynamic programming (will see this later) first, but some problems do require greedy solutions
    \end{itemize}
\end{frame}

\begin{frame}[plain]{Coin change}
    \begin{itemize}
        \item A classical example is making change. Say you want to sum up $n$ and have only denominations of $1, 5$ and $10$, what's the least amount of coins you can give back?
        \item The greedy solution would be to just always give the biggest coin you can that's not too much. So for say $24$ we'd do $10, 10, 1, 1, 1, 1$.
        \item Is this always optimal?
    \end{itemize}
\end{frame}

\begin{frame}[plain]{Coin change}
    \begin{itemize}
        \item Well, it turns out to depend on the denominations. Say we have denominations of $1, 8$ and $20$.
        \item For $n = 24$ we then give back $20, 1, 1, 1, 1$ instead of the optimal $8, 8, 8$.
        \item We will come back to this problem when we solve the general case using dynamic programming.
    \end{itemize}
\end{frame}

\begin{frame}[plain]{Taxi assignment}
    \begin{itemize}
        \item Let's consider another problem. You are managing a taxi company and today $n$ drivers showed up and you have $m$ cars. 
        \item But not all drivers and cars are created equal. Car $i$ has $h_i$ horsepower and driver $j$ can only handle at most $g_j$ horsepower.
        \item What's the greatest number of drivers you can pair to cars such that they can handle their car?
    \end{itemize}
\end{frame}

\begin{frame}[plain]{The greedy step}
    \begin{itemize}
        \item The greedy idea here is to simply pair each car to the worst driver that can still handle that car.
        \item Thus we start by sorting the drives and cars and then simply linearly walk through each and pair them together.
        \item It might not be obvious, but this actually gives the best answer.
    \end{itemize}
\end{frame}

\begin{frame}[plain,fragile]{Implementation}
\begin{minted}{cpp}
int main() {
    int n, m; cin >> n >> m;
    vi a(n), b(m);
    for(int i = 0; i < n; ++i) cin >> a[i];
    for(int i = 0; i < m; ++i) cin >> b[i];
    sort(a.begin(), a.end());
    sort(b.begin(), b.end());
    int ans = 0;
    for(int i = 0, j = 0; i < m; ++i) {
        while(j < n && a[j] < b[i]) j++;
        if(j < n) ans++, j++;
    }
    cout << ans << '\n';
}
\end{minted}
\end{frame}

\begin{frame}[plain]{Sorting}
    \begin{itemize}
        \item Greedy algorithms very often involve sorting
        \item More generally they often involve always picking the ``extremal'' option out of the local options, in some sense
        \item Biggest, shortest, cheapest, first, etc.
    \end{itemize}
\end{frame}

\begin{frame}[plain]{Job scheduling}
    \begin{itemize}
        \item Say we have a list of jobs, each starting at some time $s_j$ and finishing at some time $f_j$
        \item What's the largest amount of jobs we can complete if they can't overlap?
    \end{itemize}
\end{frame}

\begin{frame}[plain]{Solution}
    \begin{itemize}
        \item The solution is shockingly simple, but not obviously correct
        \item Order the jobs by completion time $f_j$ and then walk through them
        \item If you can complete a job in addition to the ones you've already picked, pick it
        \item The jobs you've picked by the end are the solution
    \end{itemize}
\end{frame}

\begin{frame}[plain]{Proof of correctness}
    \begin{itemize}
        \item Why is this correct though? Let's prove it.
        \item Suppose the algorithm is not optimal. Say we pick jobs of indices $i_1, i_2, \dots, i_k$ but a better solution picks $j_1, j_2, \dots, j_l$.
        \item Say the solutions agree on the first $r$ jobs (possibly $0$). 
        \item Now neither $i_{r+1}$ nor $j_{r+1}$ clash with the jobs $i_1 = j_1, i_2 = j_2, \dots, i_r = j_r$. But because we ordered things by end time, we must have that job $i_{r+1}$ ends no later than $j_{r+1}$. But then we could just as well have picked $i_{r+1}$. But this holds for any $r$, so by induction we have that $i_1, \dots, i_k$ is no worse than $j_1, \dots, j_l$, which gives a contradiction.
        \item Thus the algorithm is optimal.
    \end{itemize}
\end{frame}

\end{document}
