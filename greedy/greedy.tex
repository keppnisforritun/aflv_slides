\documentclass{beamer}
\usefonttheme[onlymath]{serif}
\usepackage[T1]{fontenc}
\usepackage[utf8]{inputenc}
\usepackage[english]{babel}
\usepackage{amsmath}
\usepackage{amssymb}
\usepackage{amsthm}
\usepackage{gensymb}
\usepackage{parskip}
\usepackage{mathtools}
\usepackage{listings}
\usepackage{hyperref}
\usepackage{graphicx}
\usepackage{color}
\usepackage{enumerate}
\usepackage{tikz}
\usetikzlibrary{calc}
\usetikzlibrary{positioning}
\usetikzlibrary{angles}
\usetikzlibrary{shapes}
\usetikzlibrary{arrows}
\usepackage{verbatim}
\usepackage{multicol}
\usepackage{array}
\usepackage{minted}
\parskip 0pt


\DeclareMathOperator{\lcm}{lcm}
\newcommand\floor[1]{\left\lfloor#1\right\rfloor}
\newcommand\ceil[1]{\left\lceil#1\right\rceil}
\newcommand\abs[1]{\left|#1\right|}
\newcommand\p[1]{\left(#1\right)}
\newcommand\sqp[1]{\left[#1\right]}
\newcommand\cp[1]{\left\{#1\right\}}
\newcommand\norm[1]{\left\lVert#1\right\rVert}
\renewcommand\Im{\operatorname{Im}}
\renewcommand\Re{\operatorname{Re}}

\usetheme{metropolis}
\definecolor{dark yellow}{rgb} {0.6,0.6,0.0}
\definecolor{dark green}{rgb} {0.0,0.6,0.0}

\graphicspath{{myndir/}}

\title{Greedy Algorithms}
\author{Atli FF}
\institute{\href{http://ru.is/td}{School of Computer Science} \\[2pt] \href{http://ru.is}{Reykjavík University}}
\titlegraphic{\hfill\includegraphics[height=0.6cm]{kattis}}

\begin{document}
\maketitle

\begin{frame}[plain]{Greedy algorithms}
    \begin{itemize}
        \item An algorithm that always makes \textit{locally} optimal moves is called greedy
        \item For some kinds of problems this will give a \textit{globally} optimal solution as well
        \item Seeing when this is the case can be very tricky, and if used in the wrong context the solution will get a WA verdict
    \end{itemize}
\end{frame}

\begin{frame}[plain]{Submitting greedy solutions}
    \begin{itemize}
        \item The tricky thing with these solutions are that it's often hard to know if you've made a mistake and thus get WA or if there's some hole in the greedy algorithm
        \item It's often easy to think of all kinds of greedy solutions, but they are very often wrong
        \item Generally one would like to consider complete search or dynamic programming (will see this later) first, but some problems do require greedy solutions
    \end{itemize}
\end{frame}

\begin{frame}[plain]{Coin change}
    \begin{itemize}
        \item A classical example is making change. Say you want to sum up $n$ and have only denominations of $1, 5$ and $10$, what's the least amount of coins you can give back?
        \item The greedy solution would be to just always give the biggest coin you can that's not too much. So for say $24$ we'd do $10, 10, 1, 1, 1, 1$.
        \item Is this always optimal?
    \end{itemize}
\end{frame}

\begin{frame}[plain]{Coin change}
    \begin{itemize}
        \item Well, it turns out to depend on the denominations. Say we have denominations of $1, 8$ and $20$.
        \item For $n = 24$ we then give back $20, 1, 1, 1, 1$ instead of the optimal $8, 8, 8$.
        \item We will come back to this problem when we solve the general case using dynamic programming.
    \end{itemize}
\end{frame}

\begin{frame}[plain]{Lilypad jump}
    \begin{itemize}
        \item Consider a frog jumping on a sequence of lily pads, there is one at $x = 0$ and one at $x = n$, with some amount of lilypads in between
        \item The frog can jump at most distance $r$
        \item When at a given lily pad, what's the best move?
    \end{itemize}
\end{frame}

\begin{frame}[plain]{Lilypad jump}
    \begin{itemize}
        \item Clearly just jump as far right as possible!
        \item But be careful, this is very contingent on the frog being able to jump any distance in $[0, r]$
        \item If it could jump any distance in $[r/2, r]$, it would not be true for example
    \end{itemize}
\end{frame}

\begin{frame}[plain]{Taxi assignment}
    \begin{itemize}
        \item Let's consider another problem. You are managing a taxi company and today $n$ drivers showed up and you have $m$ cars. 
        \item But not all drivers and cars are created equal. Car $i$ has $h_i$ horsepower and driver $j$ can only handle at most $g_j$ horsepower.
        \item What's the greatest number of drivers you can pair to cars such that they can handle their car?
    \end{itemize}
\end{frame}

\begin{frame}[plain]{The greedy step}
    \begin{itemize}
        \item The greedy idea here is to simply pair each car to the worst driver that can still handle that car.
        \item Thus we start by sorting the drives and cars and then simply linearly walk through each and pair them together.
        \item It might not be obvious, but this actually gives the best answer.
    \end{itemize}
\end{frame}

\begin{frame}[plain,fragile]{Implementation}
\begin{minted}{cpp}
int main() {
    int n, m; cin >> n >> m;
    vi a(n), b(m);
    for(int i = 0; i < n; ++i) cin >> a[i];
    for(int i = 0; i < m; ++i) cin >> b[i];
    sort(a.begin(), a.end());
    sort(b.begin(), b.end());
    int ans = 0;
    for(int i = 0, j = 0; i < m; ++i) {
        while(j < n && a[j] < b[i]) j++;
        if(j < n) ans++, j++;
    }
    cout << ans << '\n';
}
\end{minted}
\end{frame}

\begin{frame}[plain]{Sorting}
    \begin{itemize}
        \item Greedy algorithms very often involve sorting
        \item More generally they often involve always picking the ``extremal'' option out of the local options, in some sense
        \item Biggest, shortest, cheapest, first, etc.
    \end{itemize}
\end{frame}

\begin{frame}[plain]{Job scheduling}
    \begin{itemize}
        \item Say we have a list of jobs, each starting at some time $s_j$ and finishing at some time $f_j$
        \item What's the largest amount of jobs we can complete if they can't overlap?
    \end{itemize}
\end{frame}

\begin{frame}[plain]{Solution}
    \begin{itemize}
        \item The solution is shockingly simple, but not obviously correct
        \item Order the jobs by completion time $f_j$ and then walk through them
        \item If you can complete a job in addition to the ones you've already picked, pick it
        \item The jobs you've picked by the end are the solution
    \end{itemize}
\end{frame}

\begin{frame}[plain]{Proof of correctness}
    \begin{itemize}
        \item Why is this correct though? Let's prove it.
        \item Suppose the algorithm is not optimal. Say we pick jobs of indices $i_1, i_2, \dots, i_k$ but a better solution picks $j_1, j_2, \dots, j_l$.
        \item Say the solutions agree on the first $r$ jobs (possibly $0$). 
        \item Now neither $i_{r+1}$ nor $j_{r+1}$ clash with the jobs $i_1 = j_1, i_2 = j_2, \dots, i_r = j_r$. But because we ordered things by end time, we must have that job $i_{r+1}$ ends no later than $j_{r+1}$. But then we could just as well have picked $i_{r+1}$. But this holds for any $r$, so by induction we have that $i_1, \dots, i_k$ is no worse than $j_1, \dots, j_l$, which gives a contradiction.
        \item Thus the algorithm is optimal.
    \end{itemize}
\end{frame}

\begin{frame}[plain]{Many more}
    \begin{itemize}
        \item There are many many more and we will see plenty in the course
        \item Many famous algorithms are famous because they perform non-trivial greedy steps
        \item Dijkstra's algorithm, Huffman coding, Kruskal's algorithm, Horn satisfiability and many more
    \end{itemize}
\end{frame}

\begin{frame}[plain]{Proving correctness}
    \begin{itemize}
        \item How do we prove the greedy algorithms are correct?
        \item When in programming contests this is usually overkill, but at a workplace it is generally not
        \item There are two main common arguments that tackle this, but often novel methods are needed
        \item These are usually called "Greedy stays ahead" and "Exchange arguments"
    \end{itemize}
\end{frame}

\begin{frame}[plain]{Proving correctness}
    \begin{itemize}
        \item "Greedy stays ahead" aims to prove that during each greedy step it consistently stays ahead of all other possible choices
        \item "Exchange arguments" aims to show that you can turn any solution into the greedy solution with a sequence of "exchanges" without making them any worse, so the greedy solution is as good as it gets
        \item Sometimes a part of this is showing that the greedy solution outputs a valid solution at all, as that is not obvious in every case
    \end{itemize}
\end{frame}

\begin{frame}[plain]{Proving correctness}
    \begin{itemize}
        \item Consider the lilypads again.
        \item We'll now prove it's optimal, which is more work than it may seem at first
        \item First we prove that the greedy solution reaches the final lily pad if there is a path there
        \item Note again this is not true for many minor variants of the problem!
    \end{itemize}
\end{frame}

\begin{frame}[plain]{Proving correctness}
    By contradiction; suppose it did not. 
    Let the positions of the lilypads be $x_1 < x_2 < \dots < x_m$. 
    Since our algorithm didn't find a path, it must have stopped 
    at some lilypad $x_k$ and not been able to jump to a 
    future lilypad. In particular, this means it could not 
    jump to lilypad $k + 1$, so $x_k + r < x_{k+1}$.
	Since there is a path from lilypad 1 to the lilypad $m$, 
	there must be some jump in that path that starts
	before lilypad $k + 1$ and ends at or after lilypad $k + 1$.
	This jump can't be made from lilypad $k$, so it must
	have been made from lilypad $s$ for some $s < k$. But 
	then we have $x_s + r < x_k + r < x_{k+1}$, so this jump is illegal.
	We have reached a contradiction, so our assumption
	was wrong and our algorithm always finds a path.
\end{frame}

\begin{frame}[plain]{Proving correctness}
    \begin{itemize}
        \item Let's now show it actually stays ahead of any optimal solution
        \item Let $J$ be the set of jumps from our greedy algorithm and $J^*$ be an optimal set of jumps
        \item Then $|J| \geq |J^*|$ since it's optimal
        \item Let $p(i, J)$ be the position after taking the first $i$ jumps in $J$
        \item Let's prove that for all $i$ we have $p(i, J) \geq p(i, J^*)$
    \end{itemize}
\end{frame}

\begin{frame}[plain]{Proving correctness}
We proceed by induction. 
As a base case, if $i = 0$, then
$p(0, J) = 0 \geq 0 = p(0, J*)$ since the frog hasn't moved.
For the inductive step, assume that the claim holds
for some $0 \leq i < | J^*|$. We will prove the claim holds
for $i + 1$ by considering two cases:
\begin{itemize}
\item Case 1: $p(i, J) \geq p(i + 1, J^*)$. Since each jump moves forward, 
we have $p(i + 1, J) \geq p(i, J)$, so we have $p(i + 1, J) \geq p(i + 1, J^*)$.
\item Case 2: $p(i, J) < p(i + 1, J^*)$. Each jump is of size at most $r$, 
so $p(i + 1, J^*) \leq p(i, J^*) + r$. By the inductive hypothesis, we
know $p(i, J) \geq p(i, J^*)$, so $p(i + 1, J^*) \leq p(i, J) + r$. Therefore, 
the greedy algorithm can jump to position at least $p(i + 1, J^*)$. Therefore,
$p(i + 1, J) \geq p(i + 1, J^*)$.
\end{itemize}
So $p(i + 1, J) \geq p(i + 1, J^*)$, completing the induction.
\end{frame}

\begin{frame}[plain]{Proving correctness}
    \begin{itemize}
        \item And now we are almost done!
        \item Finally we just have to prove that $|J| = |J^*|$, which would mean the greedy is always optimal
    \end{itemize}
\end{frame}

\begin{frame}[plain]{Proving correctness}
Since $J^*$ is an optimal solution, we know that
$| J^*| \leq | J|$. We will prove $| J^*| \geq | J|$.
Suppose for contradiction that $| J^*| < | J|$. Let
$k = | J^*|$. From before, we have $p(k, J^*) \leq p(k, J)$.
Because the frog arrives at position $n$ after $k$ jumps
along series $J^*$, we know $n \leq p(k, J)$. Because the
greedy algorithm never jumps past position $n$, we
know $p(k, J) \leq n$, so $n = p(k, J).$ Since $| J^*| < | J|$,
the greedy algorithm must have taken another
jump after its $k$-th jump, contradicting that the
algorithm stops after reaching position $n$.
We have reached a contradiction, so our
assumption was wrong and $| J^*| = | J|$, so the
greedy algorithm produces an optimal solution.
\end{frame}

\begin{frame}[plain]{Proving correctness}
    \begin{itemize}
        \item That was a lot of effort!
        \item Now, which kind of greedy proof was that?
        \item<2-> It was a "Greedy stays ahead" proof, so let us next see an exchange proof
    \end{itemize}
\end{frame}

\begin{frame}[plain]{Proving correctness}
    \begin{itemize}
        \item To consider exchange arguments we have to do things slightly out of order and nab an algorithm from the future week of graph theory
        \item Consider a set of houses, we want to lay fibre cable between them such that each house is connected to every other house through some set of cables
        \item Doesn't have to be a direct connection, just some path exists between them
        \item For each pair of houses we are given the cost of laying a cable between them
        \item What's the cheapest cable-laying procedure?
    \end{itemize}
\end{frame}

\begin{frame}[plain]{Proving correctness}
    \begin{itemize}
        \item Our greedy algorithm will be as follows
        \item Start with just a single house and consider it "active"
        \item Choose the cheapest cable that goes between an active house and one that is not active
        \item Make the newly connected house active, and keep going
        \item Now we prove this is optimal!
    \end{itemize}
\end{frame}

\begin{frame}[plain]{Proving correctness}
    Let $T$ be the set of cables chosen by our algorithm and $T^*$ be some optimal set.
    Let $c(T)$ denote the total cost of a set of cables. We will prove $c(T) = c(T^*)$.
    If $T = T^*$ the result is obvious, so we can assume $T \neq T^*$.
    Then let $(u, v)$ be a cable in $T \setminus T^*$. Let $S$ be the set of active houses
    when $(u, v)$ was added to $T$ and $H$ be the set of all houses. Then $(u, v)$ is the cheapest
    edge between $S$ and $H \setminus S$. Since $T^*$ connects all houses it must contain some
    path from $u$ to $v$. This path begins in $S$ and ends in $H \setminus S$ so there must be some
    cable $(x, y)$ such that $x \in S$ and $y \in H \setminus S$. Since $(u, v)$ is the cheapest such
    edge we must have $c(\{(u, v)\}) \leq c(\{(x, y)\})$. Let $T' = T^* \cup \{(u, v)\} \setminus \{(x, y)\}$.
\end{frame}

\begin{frame}[plain]{Proving correctness}
    Since every house in $S$ can reach every other house in $S$ without using $(u, v)$, $T'$ is valid for
    those houses, and the same goes for $H \setminus S$. But then $T'$ allows any house in $S$ to reach
    $u$, then go to $v$, then to any house in $H \setminus S$. So $T'$ is a valid set of connections. But note
    that $c(T') = c(T^*)  - c(\{(x, y)\}) + c(\{(u, v)\}) \leq c(T^*)$. Since $T^*$ is optimal this means
    $c(T') \geq c(T^*)$, so $c(T') = c(T^*)$. Note that $|T \setminus T'| = |T \setminus T^*| - 1$, so if
    we repeat this same argument once for each edge in $T \setminus T^*$ we will have converted
    $T^*$ into $T$ without changing $T$, thus $c(T) = c(T^*)$.
\end{frame}

\begin{frame}[plain]{Proving correctness}
    \begin{itemize}
        \item Very mathy!
        \item But that is an example of an "Exchange argument" proof of correctness
    \end{itemize}
\end{frame}

\end{document}
