\documentclass{beamer}
\usefonttheme[onlymath]{serif}
\usepackage[T1]{fontenc}
\usepackage[utf8]{inputenc}
\usepackage[english]{babel}
\usepackage{amsmath}
\usepackage{amssymb}
\usepackage{amsthm}
\usepackage{gensymb}
\usepackage{parskip}
\usepackage{mathtools}
\usepackage{listings}
\usepackage{hyperref}
\usepackage{graphicx}
\usepackage{color}
\usepackage{enumerate}
\usepackage{tikz}
\usetikzlibrary{calc}
\usetikzlibrary{positioning}
\usetikzlibrary{angles}
\usetikzlibrary{shapes}
\usetikzlibrary{arrows}
\usepackage{verbatim}
\usepackage{multicol}
\usepackage{array}
\usepackage{minted}
\parskip 0pt


\DeclareMathOperator{\lcm}{lcm}
\newcommand\floor[1]{\left\lfloor#1\right\rfloor}
\newcommand\ceil[1]{\left\lceil#1\right\rceil}
\newcommand\abs[1]{\left|#1\right|}
\newcommand\p[1]{\left(#1\right)}
\newcommand\sqp[1]{\left[#1\right]}
\newcommand\cp[1]{\left\{#1\right\}}
\newcommand\norm[1]{\left\lVert#1\right\rVert}
\renewcommand\Im{\operatorname{Im}}
\renewcommand\Re{\operatorname{Re}}

\usetheme{metropolis}
\definecolor{dark yellow}{rgb} {0.6,0.6,0.0}
\definecolor{dark green}{rgb} {0.0,0.6,0.0}

\graphicspath{{myndir/}}

\title{Ad Hoc}
\author{Arnar Bjarni Arnarson \& Atli FF}
\institute{\href{http://ru.is/td}{School of Computer Science} \\[2pt] \href{http://ru.is}{Reykjavík University}}
\titlegraphic{\hfill\includegraphics[height=0.6cm]{kattis}}

\begin{document}
\maketitle

\begin{frame}[plain]{Ad Hoc}
    \begin{itemize}
        \item Not much to say about ad hoc problems
        \item Literally means ``for this'' in Latin
        \item No general solution strategy used
        \item Some people tend to classify pure implementation problem as ad hoc
        \item Ad hoc means you should do what the statement says
        \item No tricks, just work
        \item Does \textit{not} mean they are easy
        \item Read statements carefully
    \end{itemize}
\end{frame}

\begin{frame}[plain]{Examples}
    \begin{itemize}
        \item https://ru.kattis.com/problems/asciikassi3
        \item https://ru.kattis.com/problems/deildadrottnun
        \item https://ru.kattis.com/problems/vasaljos
        \item https://ru.kattis.com/problems/deathknight
        \item https://ru.kattis.com/problems/vote
    \end{itemize}
\end{frame}

\end{document}

