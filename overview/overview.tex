\documentclass{beamer}
\usefonttheme[onlymath]{serif}
\usepackage[T1]{fontenc}
\usepackage[utf8]{inputenc}
\usepackage[english]{babel}
\usepackage{amsmath}
\usepackage{amssymb}
\usepackage{amsthm}
\usepackage{gensymb}
\usepackage{parskip}
\usepackage{mathtools}
\usepackage{listings}
\usepackage{hyperref}
\usepackage{color}
\usepackage{enumerate}

\usepackage{verbatim}
\usepackage{minted}
\parskip 0pt


\DeclareMathOperator{\lcm}{lcm}
\newcommand\floor[1]{\left\lfloor#1\right\rfloor}
\newcommand\ceil[1]{\left\lceil#1\right\rceil}
\newcommand\abs[1]{\left|#1\right|}
\newcommand\p[1]{\left(#1\right)}
\newcommand\sqp[1]{\left[#1\right]}
\newcommand\cp[1]{\left\{#1\right\}}
\newcommand\norm[1]{\left\lVert#1\right\rVert}
\renewcommand\Im{\operatorname{Im}}
\renewcommand\Re{\operatorname{Re}}

\usetheme{metropolis}
\definecolor{dark yellow}{rgb} {0.6,0.6,0.0}
\definecolor{dark green}{rgb} {0.0,0.6,0.0}

\graphicspath{{myndir/}}

\title{Introduction}
\author{Atli FF}
\institute{\href{http://ru.is/td}{School of Computer Science} \\[2pt] \href{http://ru.is}{Reykjavík University}}
\titlegraphic{\hfill\includegraphics[height=0.6cm]{kattis}}
\date{\textbf{Árangursrík forritun og lausn verkefna}}

\begin{document}

\begin{frame}[plain]
    \titlepage
\end{frame}

\section*{Course Overview}

\begin{frame}[plain]
	\frametitle{Welcome}
	\begin{itemize}
		 \item T-414-AFLV, Árangursrík forritun og lausn verkefna
         \vspace{10pt}
         \item Atli FF, {\alert{atlifannar@hi.is}, \alert{atlif@ru.is}}
         \item Arnar Bjarni Arnarson, {\alert{arnarar@ru.is}}
	\end{itemize}
\end{frame}

\begin{frame}[plain]
	\frametitle{Learning goals}
	\begin{itemize}
		 \item At its heart this course is about problem solving.
		 \item In this course you will learn to take a problem and:
		 \begin{itemize}
		 	\item analyse the constraints of the problem,
		 	\item take those and find applicable algorithms and data structures,
		 	\item convert those ideas into a functional program,
		 	\item do this quickly and under pressure,
		 	\item producing a program without bugs or other errors.
		 \end{itemize}
	\end{itemize}
\end{frame}

\begin{frame}[plain]
	\frametitle{Getting there}
	\begin{itemize}
		 \item We will get to this point by going over a number of things:
		 \begin{itemize}
		 	\item look at common problem types,
		 	\item cover different kinds of problem solving paradigms,
		 	\item show common algorithms and data structures you should know already in action,
		 	\item introduce new less common algorithms and data structures,
		 	\item go over useful theories from a few branches of mathematics,
		 	\item practice solving problems,
		 	\item practice more,
		 	\item practice more,
		 	\item and practice more!
		 \end{itemize}
	\end{itemize}
\end{frame}

\begin{frame}[plain]
	\frametitle{Teaching material}
	\begin{itemize}
		 \item This course will loosely follow \alert{Competitive Programming} by Steven Halim
		 \item First edition can be downloaded from the book homepage \textbf{https://cpbook.net/}
		 \item A different set of slides for additional reading can also be found at \textbf{https://github.com/Kakalinn/tol607g-glaerur}
         \item Other good material can be found on \textbf{cp-algorithms.com}, \textbf{codeforces.com}, \textbf{wiki.algo.is} and more
         \item There are plenty of links on Canvas!
     \end{itemize}
\end{frame}

\begin{frame}[plain]
	\frametitle{Piazza}
	\begin{itemize}
		\item Piazza can be used to ask questions
        \item Link is on canvas
        \item Before posting questions, read the pinned announcement on what questions can be made publically
	\end{itemize}
\end{frame}

\begin{frame}[plain]
	\frametitle{Course Schedule}
	\scriptsize
    \begin{center}
        \begin{tabular}{cl|ll}
            Week no. & Date & Topic \\ \hline
            1 & 18.08 & Complexity and Standard Libraries \\
            2 & 25.08 & Ad-hoc and Complete Search \\
            3 & 01.09 & Greedy and Reduce \& Conquer \\
            4 & 08.09 & Divide \& Conquer and Dynamic Programming 1 \\
            & 13.09 & FKHI (10-15 GMT) \\
            5 & 15.09 & Dynamic Programming 2 \\
            6 & 22.09 & Data Structures \\
            7 & 29.09 & Math and Number Theory \\
            & 04.10 & NCPC (9-14 GMT) \\
            8 & 06.10 & Graphs 1 \\
            9 & 13.10 & Graphs 2 \\
            10 & 20.10 & Graphs 3 \\
            11 & 27.10 & Combinatorics \& Dynamic Programming 3 \\
            12 & 03.11 & Strings \\
             & 29.11 & NWERC \\
        \end{tabular}
    \end{center}
\end{frame}

\begin{frame}[plain]
	\frametitle{Problem Sets}
	\begin{itemize}
		\item Each week covers a particular topic
        \item Groups of up to three people can discuss the problems, but each individual must write and hand in their own code
        \item We will check for similar submissions, and take action if we think that people are cheating
        \item Kattis also has a built in anti-cheat feature, which in my personal experience has been plenty good enough to catch a lot of cheaters
	\end{itemize}
\end{frame}

\begin{frame}[plain]
	\frametitle{Problem Sets cntd.}
	\begin{itemize}
	    \item Each problem set has $\sim 6$ problems
        \item To get a perfect score you need to get at least a certain amount of points
        \item The problems will give varying number of points depending on difficulty
        \item The grade is not linear, getting half of the perfect score gets you $7.5$
        \item The deadline for problem sets is always the week's Sunday
        \item Late handins will not be accepted
        \item Kattis' verdict is law
	\end{itemize}
\end{frame}

\begin{frame}[plain]
	\frametitle{Bonus problems}
	\begin{itemize}
	    \item Each problem set contains 2 challenging bonus problems
        \item Deadline for all bonus problems is the same as the deadline for the last problem set
        \item Bonus problems are only included in the final grade if the student passes the course before inclusion
        \item These can raise the final grade by up to $10\%$
	\end{itemize}
\end{frame}

\begin{frame}[plain]
	\frametitle{Other bonuses}
	\begin{itemize}
	    \item Participating in FKHI and/or NCPC will count as a submitted problem set
	    \item This can effectively replace the lowest or two lowest problem set grades
	    \item Good class/piazza participation can also be rewarded per teacher's discretion
	\end{itemize}
\end{frame}

\begin{frame}[plain]
	\frametitle{Evaluation}
	\begin{center}
        \begin{tabular}{lr}
            Problem sets & $60\%$ \\
            Midterm exam & $20\%$ \\
            Final exam & $20\%$ \\
            Bonus problems/participation & $10\%$ \\
            \hline
            Total & $110\%$ \\
        \end{tabular}
	\end{center}
	
	\begin{itemize}
        \item Remember that bonus problems are only considered if the student passes the course, and is only used to raise the final grade
        \item A final grade greater than 10 will be reduced down to 10
        \item The final exam and midterm exam must be passed to pass the course
	\end{itemize}
\end{frame}


\section*{Problem structure and Kattis}

\begin{frame}[plain]
	\frametitle{Problem Structure}
	\begin{itemize}
		\item A typical programming contest problem usually consists of a
        \begin{itemize} 
            \item Problem description
            \item Input description
            \item Output description
            \item Example input/output
            \item A time limit in seconds
            \item A memory limit in bytes
        \end{itemize}
        \item You are asked to write a program that solves the problem for all valid inputs
        \item The program must not exceed time or memory limits
	\end{itemize}
\end{frame}

\begin{frame}[plain]
	\frametitle{Example Problem}
	\begin{block}{Problem description}
    		Write a program that multiplies pairs of integers.
    \end{block}

    \vspace{10pt}
    
    \begin{block}{Input description}
    		Input starts with one line containing an integer $T$, where $1\leq T \leq
    100$, denoting the number of test cases. Then $T$ lines follow, each
    containing a test case. Each test case consists of two integers $A,B$,
    where $-2^{20} \leq A,B \leq 2^{20}$, separated by a single space.
    \end{block}

    \vspace{10pt}
    
    \begin{block}{Output description}
    		For each test case, output one line containing the value of $A\times B$.
    \end{block}
\end{frame}

\begin{frame}[plain]
	\frametitle{Example Problem}
	\begin{center}
		\begin{tabular}{|l|l|}
            \hline
            {\footnotesize Sample input} & {\footnotesize Sample output} \\
            \hline
            \ttfamily
            4 &  \\
            3 4 & 12 \\
            13 0 & 0 \\
            1 8 & 8 \\
            100 100 & 10000 \\
            \hline
        \end{tabular}
    \end{center}
\end{frame}

\begin{frame}[plain, fragile]
    \frametitle{Possible Solution}
	\begin{scriptsize}
        \begin{minted}{cpp}
#include <iostream>
using namespace std;

int main() {
    int32_t T;
    cin >> T;
    for(int32_t t = 0; t < T; t++) {
        int32_t A, B;
        cin >> A >> B;
        cout << A * B << endl;
    }
    return 0;
}
        \end{minted}
    \end{scriptsize}
    \begin{itemize}
       \item<2-> Is this correct? \only<5-> {\alert{No!}}
       \item<3-> What if $A = B = 2^{20}$? \only<4-> {The output is $0$ }
    \end{itemize}
\end{frame}

\begin{frame}[plain, fragile]
    \frametitle{Fixed Solution}
	\begin{scriptsize}
        \begin{minted}{cpp}
#include <iostream>
using namespace std;

int main() {
    int32_t T;
    cin >> T;
    for(int32_t t = 0; t < T; t++) {
        int64_t A, B;
        cin >> A >> B;
        cout << A * B << endl;
    }
    return 0;
}
        \end{minted}
    \end{scriptsize}
    \begin{itemize}
       \item<2-> Is this correct? \only<4-> {\alert{Yes!}}
       \item<3-> The values are at most $2^{20}$ in absolute value, so their product is at most $2^{40}$ in absolute value, which fits. 
    \end{itemize}
\end{frame}

\begin{frame}[plain]
    \frametitle{Automatic Judging}
    \begin{itemize}
        \item The problems will be available on \alert{Kattis}:
        \item \textbf{https://ru.kattis.com/}
        \vspace{20pt}
        \item Kattis is an online judge
        \item You will submit your solutions to Kattis, and get immediate feedback about the solution
        \item You can submit in any of the supported languages:
        \begin{itemize}
            \item C, C++, Java, Python, C\#{}, Javascript
            \item and \textbf{many} others
        \end{itemize}
    \end{itemize}
\end{frame}

\begin{frame}[plain]
    \frametitle{Verdicts}
    \begin{itemize}
        \item Feedback is (intentionally) limited
        \item You will (almost always) receive one of the following verdicts:
        \begin{itemize}
            \item Accepted (AC)
            \item Wrong Answer (WA)
            \item Compile Error (CE)
            \item Run Time Error (RTE)
            \item Time Limit Exceeded (TLE)
            \item Memory Limit Exceeded (MLE)
        \end{itemize}
        \item Neither we nor Kattis will give away info on the test data used to test solutions
    \end{itemize}
\end{frame}


\end{document}

