\documentclass{beamer}
\usefonttheme[onlymath]{serif}
\usepackage[T1]{fontenc}
\usepackage[utf8]{inputenc}
\usepackage[english]{babel}
\usepackage{amsmath}
\usepackage{amssymb}
\usepackage{amsthm}
\usepackage{gensymb}
\usepackage{parskip}
\usepackage{mathtools}
\usepackage{listings}
\usepackage{hyperref}
\usepackage{graphicx}
\usepackage{color}
\usepackage{enumerate}
\usepackage{tikz}
\usetikzlibrary{calc}
\usetikzlibrary{positioning}
\usetikzlibrary{angles}
\usetikzlibrary{shapes}
\usetikzlibrary{arrows}
\usepackage{verbatim}
\usepackage{multicol}
\usepackage{array}
\usepackage{minted}
\parskip 0pt


\DeclareMathOperator{\lcm}{lcm}
\newcommand\floor[1]{\left\lfloor#1\right\rfloor}
\newcommand\ceil[1]{\left\lceil#1\right\rceil}
\newcommand\abs[1]{\left|#1\right|}
\newcommand\p[1]{\left(#1\right)}
\newcommand\sqp[1]{\left[#1\right]}
\newcommand\cp[1]{\left\{#1\right\}}
\newcommand\norm[1]{\left\lVert#1\right\rVert}
\renewcommand\Im{\operatorname{Im}}
\renewcommand\Re{\operatorname{Re}}

\usetheme{metropolis}
\definecolor{dark yellow}{rgb} {0.6,0.6,0.0}
\definecolor{dark green}{rgb} {0.0,0.6,0.0}

\graphicspath{{myndir/}}

\title{Square Root Decomposition}
\author{Arnar Bjarni Arnarson}
\institute{\href{http://ru.is/td}{School of Computer Science} \\[2pt] \href{http://ru.is}{Reykjavík University}}
\titlegraphic{\hfill\includegraphics[height=0.6cm]{kattis}}

\begin{document}
\maketitle

\begin{frame}[plain]{Range queries}
    \vspace{30pt}
    \begin{itemize}
        \item<1-> We have an array $A$ of size $n$.
        \item<2-> Given $i,j$, we want to answer:
            \begin{itemize}
                \item<3-> $\mathrm{max}(A[i],A[i+1],\ldots,A[j-1],A[j])$
                \item<4-> $\mathrm{min}(A[i],A[i+1],\ldots,A[j-1],A[j])$
                \item<5-> $\mathrm{sum}(A[i],A[i+1],\ldots,A[j-1],A[j])$
            \end{itemize}
        \item<6-> We want to answer these queries efficiently, or in other words, without looking through all elements.
        \item<7-> Sometimes we also want to update elements.
    \end{itemize}
\end{frame}

\begin{frame}[plain]{Bucketing}
    \begin{itemize}
        \item<1-> Group values into buckets of size $k$ and store result of each bucket
        \item<2-> Updating is easy:
        \begin{itemize}
            \item<3-> change the array element
            \item<4-> recompute corresponding bucket
        \end{itemize}
        \item<5-> Time complexity: $O(k)$
        \item<6-> Again we want to query over a range
        \begin{itemize}
            \item<7-> When a bucket is contained in the range, use the stored sum for the bucket
            \item<8-> This (sometimes) allows us to ``jump'' over intervals of size $k$
            \item<9-> Only have to go inside at most two buckets (each end)
            \item<10-> Have to consider at most $n/k$ buckets and 2 buckets of size $k$
        \end{itemize}
        \item<11-> Time complexity: $O(n/k + k)$
    \end{itemize}
\end{frame}

\begin{frame}[plain]{Choosing number of buckets}
    \begin{itemize}
        \item<1-> Now we have a data structure that supports:
            \begin{itemize}
                \item<2-> Updating in $O(k)$
                \item Querying in $O(n/k + k)$
            \end{itemize}
        \item<3-> What $k$ to pick?
        \item<4-> Time complexity is minimized for $k=\sqrt{n}$:
            \begin{itemize}
                \item<5-> Updating in $O(\sqrt{n})$
                \item<6-> Querying in $O(n/\sqrt{n} + \sqrt{n}) = O(\sqrt{n})$
            \end{itemize}
        \item<7-> Also known as square root decomposition, and is a very
            powerful technique
    \end{itemize}
\end{frame}

\begin{frame}[plain]{Example problem: Supercomputer}
    \begin{itemize}
        \item https://open.kattis.com/problems/supercomputer
    \end{itemize}
\end{frame}

\begin{frame}[plain]{Mo's Algorithm}


\end{frame}

\end{document}
