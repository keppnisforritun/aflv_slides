\documentclass{beamer}
\usefonttheme[onlymath]{serif}
\usepackage[T1]{fontenc}
\usepackage[utf8]{inputenc}
\usepackage[english]{babel}
\usepackage{amsmath}
\usepackage{amssymb}
\usepackage{amsthm}
\usepackage{gensymb}
\usepackage{parskip}
\usepackage{mathtools}
\usepackage{listings}
\usepackage{hyperref}
\usepackage{graphicx}
\usepackage{color}
\usepackage{enumerate}
\usepackage{tikz}
\usetikzlibrary{calc}
\usetikzlibrary{positioning}
\usetikzlibrary{angles}
\usetikzlibrary{shapes}
\usetikzlibrary{arrows}
\usepackage{verbatim}
\usepackage{multicol}
\usepackage{array}
\usepackage{minted}
\parskip 0pt


\DeclareMathOperator{\lcm}{lcm}
\newcommand\floor[1]{\left\lfloor#1\right\rfloor}
\newcommand\ceil[1]{\left\lceil#1\right\rceil}
\newcommand\abs[1]{\left|#1\right|}
\newcommand\p[1]{\left(#1\right)}
\newcommand\sqp[1]{\left[#1\right]}
\newcommand\cp[1]{\left\{#1\right\}}
\newcommand\norm[1]{\left\lVert#1\right\rVert}
\renewcommand\Im{\operatorname{Im}}
\renewcommand\Re{\operatorname{Re}}

\usetheme{metropolis}
\definecolor{dark yellow}{rgb} {0.6,0.6,0.0}
\definecolor{dark green}{rgb} {0.0,0.6,0.0}

\graphicspath{{myndir/}}

\title{Union-Find}
\author{Arnar Bjarni Arnarson}
\institute{\href{http://ru.is/td}{School of Computer Science} \\[2pt] \href{http://ru.is}{Reykjavík University}}
\titlegraphic{\hfill\includegraphics[height=0.6cm]{kattis}}

\begin{document}
\maketitle

\begin{frame}[plain]{Union-Find}
    \begin{itemize}
        \item<1-> We have $n$ items
        \item<1-> Maintains a collection of disjoint sets (or equivalently, an equivalence relation)
        \item<1-> Each of the $n$ items is in exactly one set
        \item<2-> We represent each set with one of its members, a representative element
        \item<2-> Supports two operations efficiently: \texttt{find(x)} and \texttt{union(x,y)}.
        \item<3-> Operation \texttt{find(x)} finds the representative of the set $x$ is in
        \item<4-> Operation \texttt{union(x, y)} unions the sets of which $x$ and $y$ are members.
    \end{itemize}
\end{frame}

\begin{frame}[plain]{Union-Find}
    \begin{itemize}
        \item<1-> It is generally initialized with all items being in their own set.
        \item<1-> So for $n = 5$ we start out with $\{\{1\},\{2\},\{3\},\{4\},\{5\}\}$.
        \item<2-> \texttt{union(1, 3)} then changes this to $\{\{1, 3\}, \{2\}, \{4\}, \{5\}\}$.
		\item<2-> \texttt{union(2, 5)} then results in $\{\{1, 3\}, \{2, 5\}, \{4\}\}$.
		\item<2-> \texttt{union(2, 4)} then results in $\{\{1, 3\}, \{2, 4, 5\}\}$.
		\item<2-> \texttt{union(1, 4)} finally results in $\{\{1, 2, 3, 4, 5\}\}$.
        \item<3-> At any given point $\texttt{find(x)}$ returns some value in the same set as $x$.
        \item<3-> The important bit is that \texttt{find(x)} returns the same value for all elements of the same set, the representative.
    \end{itemize}
\end{frame}

\begin{frame}[plain]{Union-Find}
    \begin{itemize}
        \item<1-> We can do this by maintaining an array of parents, letting the $i$-th value be the index of the parent of the $i$-th item.
        \item<1-> If a value has no parent, we can denote this somehow, make it its own parent, give it the value $-1$, exactly what we do is not important.
        \item<2-> To get the representative of $x$ we go to the parent of our current item (starting at $x$) until the item has no parent.
        \item<2-> Then to unite $x, y$ we simply make the representative of $x$ the parent of the representative of $y$.
    \end{itemize}
\end{frame}

\begin{frame}[plain,fragile]{Quick-Find}
    \begin{small}
    \begin{minted}{cpp}
struct union_find {
    vector<int> parent;
    union_find(int n) {
        parent = vector<int>(n);
        for(int i = 0; i < n; i++) {
            parent[i] = i;
        }
    }
    int find(int x) {
        return parent[x];
    }
    void unite(int x, int y) {
        for (int i = 0; i < n; i++) {
            if (find(i) == find(x)) {
                parent[i] = find(y);
            }
        }
    }
};
    \end{minted}
    \end{small}
\end{frame}

\begin{frame}
    \begin{itemize}
        \item The time complexity of the find operation is $\mathcal{O}(1)$.
        \item The time complexity of the unite operation is $\mathcal{O}(n)$.
        \item This is not good enough, so lets try a different approach.
    \end{itemize}
\end{frame}

\begin{frame}[plain,fragile]{Quick-Union}
    \begin{minted}{cpp}
struct union_find {
    vector<int> parent;
    union_find(int n) {
        parent = vector<int>(n);
        for(int i = 0; i < n; i++) {
            parent[i] = i;
        }
    }
    int find(int x) {
        return parent[x] == x ? x : find(parent[x]);
    }
    void unite(int x, int y) {
        parent[find(x)] = find(y);
    }
};
    \end{minted}
\end{frame}

\begin{frame}[plain]{Union-Find}
    \begin{itemize}
        \item<1-> The problem here is that for specific queries, the parent chains may end up being of length $\mathcal{O}(n)$, making each query linear.
        \item<1-> The key to making this more efficient is making those chains shorter.
        \item<2-> One method is to use what is known as small-to-large merging, where the smaller group's leader is made to point to the larger group's leader.
        \item<2-> This ensures the height increases by $1$ as a group's size doubles, resulting in $\mathcal{O}(\log n)$ complexity.
        \item<3-> We can also do this by flattening the chain each time we query \texttt{find}, so the amortized complexity becomes good.
        \item<3-> Here the worst case is still $\mathcal{O}(n)$ but the amortized complexity is $\mathcal{O}(\alpha(n))$ which may as well be a constant, as it is $<5$ for $n$ equal to the number of atoms in the observable universe.
    \end{itemize}
\end{frame}

\begin{frame}[plain,fragile]{Weighted Quick-Union}
    \begin{tiny}
    \begin{minted}{cpp}
struct union_find {
    vector<int> parent, sizes;
    union_find(int n) : parent(n), sizes(n) {
        for (int i = 0; i < n; i++) {
            parent[i] = i;
            sizes[i] = 1;
        }
    }
    int find(int x) {
        if(parent[x] == x) return x;
        return find(parent[x]);
    }
    void unite(int x, int y) {
        int xp = find(x);
        int yp = find(y);
        if (xp == yp) return;
        if (sizes[xp] > sizes[yp]) swap(xp, yp);
        parent[xp] = parent[yp];
        sizes[yp] += sizes[xp];
    }
};
    \end{minted}
    \end{tiny}
\end{frame}

\begin{frame}[plain,fragile]{Quick-Union with path compression}
    \begin{small}
    \begin{minted}{cpp}
struct union_find {
    vector<int> parent;
    union_find(int n) {
        parent = vector<int>(n);
        for (int i = 0; i < n; i++) {
            parent[i] = i;
        }
    }
    int find(int x) {
        if(parent[x] == x) return x;
        return parent[x] = find(parent[x]);
    }
    void unite(int x, int y) {
        parent[find(x)] = find(y);
    }
};
    \end{minted}
    \end{small}
\end{frame}

\begin{frame}[plain]{Union-Find applications}
    \vspace{30pt}
    \begin{itemize}
        \item<1-> Union-Find maintains a collection of disjoint sets
        \item<1-> When are we dealing with such collections?
        \item<2-> Usually when we want to work with equivalence relations like graph connectivity
        \item<2-> By modifying the data structure it can also contain more queryable data
        \begin{itemize}
            \item<2-> Number of different sets currently
            \item<2-> Current size of the set containing $x$
            \item<2-> An iterable list of all elements of the set containing $x$
        \end{itemize}
        \item<3-> When tracking size you can use it to always perform small-to-large merges for $\mathcal{O}(\log n)$ time complexity.
    \end{itemize}
\end{frame}

\begin{frame}[plain]{Example problem: Skolavslutningen}
    \begin{itemize}
        \item https://open.kattis.com/problems/skolavslutningen
    \end{itemize}
\end{frame}


\end{document}
